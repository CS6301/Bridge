\documentclass{article}

\usepackage{cmbright}
\usepackage{float}
\usepackage{amsmath}
\usepackage{mathtools}
\usepackage{hyperref}
\usepackage{cleveref}
\usepackage{algorithm}
\usepackage{algpseudocode}

\title{Algorithms Design for Bridge Problem}
\author{Hanlin He\footnote{\texttt{hxh160630@utdallas.edu}},
Lizhong Zhang\footnote{\texttt{lxz160730@utdallas.edu}}}
\date{\today}

\begin{document}

\maketitle

\section*{Semaphores Algorithm}

The solution using semaphores is shown in \cref{semaphore1}.

\begin{algorithm}[H]
\caption{Attempt Semaphores Solution}\label{semaphore1}
\begin{algorithmic}
    \State{int[2] $count \leftarrow \{0, 0\}$}
    \State{semaphore[2] $mutex \leftarrow \{1, 1\}$}
    \State{semaphore $service \leftarrow 1$}
    \State{semaphore $resource \leftarrow 1$}
    \State{}
    \Function{arriveBridge}{$direction$}\Comment{Assume $direction$ is $0/1$}
        \State{\Call{down}{$service$}}
        \State{\Call{down}{$mutex[direction]$}}
        \State{$count[direction] \leftarrow count[direction] + 1$}
        \If{$count[direction] = 1$}\Comment{First one need to acquire $resource$}
            \State{\Call{down}{$resource$}}
        \EndIf{}
        \State{\Call{up}{$mutex[direction]$}}
        \State\Call{up}{$service$}
    \EndFunction{}
    \State{}
    \Function{leaveBridge}{$direction$}
        \State{\Call{down}{$mutex[direction]$}}
        \State{$count[direction] \leftarrow count[direction] - 1$}
        \If{$count[direction] = 0$}\Comment{Last one need to release $resource$}
            \State{\Call{up}{resource}}
        \EndIf{}
        \State{\Call{up}{$mutex[direction]$}}
    \EndFunction{}
\end{algorithmic}
\end{algorithm}

\subsection*{Analysis}

This algorithm mimic the reader logic in the `Fair Readers-Writers Solution'.
It uses array of integers/semaphores for two directions and direction as the
reference index into the array. Each $mutex[i]$ ensure exclusive read/write
operation on each $count[i]$. Traffics from one direction can enter the bridge
without waiting if no traffic was waiting on the other side (just like any
number of readers can read if no writer was waiting). And as long as there is
some traffics waiting on the other side, the after traffic will wait until the
other end traffic enter bridge once.

But this algorithm relies on the implementation of semaphores to ensure that
all cars will come one direction after the other if both directions have
traffic waiting. Consider scenario as follow:
\begin{enumerate}
    \item Process A first called \verb|arriveBridge| from direction $0$.
    \item Before A return,
        \begin{enumerate}
            \item Process B called \verb|arriveBridge| from direction $1$.
            \item Process C called \verb|arriveBridge| from direction $0$.
        \end{enumerate}
    \item Process A return from \verb|arriveBridge|.
\end{enumerate}
By requirement, the next process `cross the bridge' should be B. In the
algorithm, both B and C are `spinning' on \verb|down(service)|. After A invoke
\verb|leaveBridge| and release $service$, it depends on the implementation of
the semaphore to determine which process among B and C could acquire the
$service$. It's possible one side traffic got overtaken by the other side any
number of times, but eventually the traffic will leave the bridge.

\pagebreak

\section*{Monitors Algorithm}

The solution using monitors is shown in \cref{monitor}.

\begin{algorithm}[H]
\caption{Attempt Monitors Solution}\label{monitor}
\begin{algorithmic}
    \State{int $active[2] \leftarrow \{0, 0\}$}
    \State{int $waiting[2] \leftarrow \{0, 0\}$}
    \State{condition $canPass[2]$}
    \State{}
    \Function{arriveBridge}{$direction$}\Comment{Assume $direction$ is $0/1$}
        \State{$other \leftarrow 1 - direction$}
        \If{$active[other] + waiting[other] > 0$}
            \State{$waiting[direction] \leftarrow waiting[direction] + 1$}
            \State{\Call{wait}{$canPass[direction]$}}
            \State{$waiting[direction] \leftarrow waiting[direction] - 1$}
        \EndIf{}
        \State{$active[direction] \leftarrow active[direction] + 1$}
    \EndFunction{}
    \State{}
    \Function{leaveBridge}{$direction$}
        \State{$other \leftarrow 1 - direction$}
        \State{$active[direction] \leftarrow active[direction] - 1$}
        \If{$active[direction] = 0 \land waiting[other] > 0$}
            \If{$waiting[direction] > 0$}
                \State{\Call{signal}{$canPass[other]$}}
            \Else
                \State{\Call{signalAll}{$canPass[other]$}}
            \EndIf{}
        \EndIf{}
    \EndFunction{}
\end{algorithmic}
\end{algorithm}

\subsection*{Analysis}

This algorithm mimic the reader logic in the `Starvation Free Readers-Writers
Solution with Monitors'.

During \verb|leaveBridge|, if current car is the last of its direction to leave
the bridge for the moment, and there are cars waiting on the other side,
prepare to signal the other end to pass. And if current car's direction also
has cars waiting, then only signal one car from the other end to enter,
otherwise, signal all cars on the other direction to enter.

\end{document}
